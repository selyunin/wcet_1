\label{problem:3}

First, focus on the analysis of \texttt{task.c:merge\_samples()}.  The loop
bounds in this function depend on the input-data dependent parameter
\texttt{@inputcount}, defined to be the maximal value of
\texttt{input->input\_count}. Add loop bounds and flow facts for
\texttt{merge\_samples}, and analyze its WCET assuming that
$\texttt{@inputcount} \leq 64$.

\begin{itemize}

\item[Q1:]
  Describe and justify the loop bounds and flow facts used in the WCET analysis
  of \texttt{merge\_samples}.

\item[Q2:]
  What is the maximum execution time observed for this function? What is the
  analyzed WCET? In case they do not coincide, discuss reasons for the
  impreciseness in the static analysis.

\item[Hint:]
  For measurements, you might want to rely on or start with the test bench
  implemented in\\
  \texttt{main.c:test\_merge\_samples()}.

\item[Hint:]
  Division is implemented in software for the Patmos architecture. We replaced
  the division in the interpolation routine by a lookup table for small values,
  to simplify the analysis problem. You do not need to worry about the analysis
  of \texttt{iinterpolate16}; simply use the following annotations:

\end{itemize}

\begin{verbatim}
  include "llvm.ais";
  instruction "iinterpolate16" + 1 calls is never executed;
  instruction "iinterpolate16" + 2 calls is never executed;
\end{verbatim}
